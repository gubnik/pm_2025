\documentclass[14pt]{extarticle}

\usepackage{fontspec}
\usepackage{geometry}
\geometry{a4paper, left=30mm, right=10mm, top=20mm, bottom=20mm}
\usepackage[hidelinks]{hyperref}
\usepackage{polyglossia}
\usepackage{titlesec}
\usepackage{pgfplots}
\usepackage{amsmath}
\usepackage{amssymb}
\usepackage{array}
\usepackage{tikz}
\usepackage{pgf}
\usepackage{xcolor}
\usepackage{breqn}
\usepackage{natbib}
\usepackage{enumitem}
\usepackage{fancyhdr}
\usepackage{footmisc}
\usepackage{unicode-math}
\usepackage{gensymb}

\usepackage{setspace}
\onehalfspacing

\setlength{\parindent}{12.7mm}

\usepackage{ragged2e}
\raggedbottom
\sloppy
\justifying

\pagestyle{fancy}
\bibliographystyle{plain}
\setmainfont{Times New Roman}[size=14pt]
\setdefaultlanguage{russian}
\usepackage[utf8]{inputenc}

\begin{document}
\include{gen.titlepage}
\tableofcontents
\fancyhead[L]{\textit{\leftmark}}
\fancyhead[R]{}
\fancyfoot[C]{\thepage}
\clearpage

\section{Аннотация}
\paragraph{Цель работы.} Разработать портативный поршневой насос полевого назначения, обеспечивающий подачу воды
с обеспечением параметров расхода и напора, соответствующим требованиям ГОСТ, с минимизацией пульсации
\footnote{Пульсация -- периодические колебания скорости или расхода жидкости, проходящий через элемент гидравлической системы}
потока и обеспечением безопасности эксплуатации.

\paragraph{Задачи проекта.}
\begin{enumerate}
    \item Проведение метрического синтеза и кинематического анализа кривошипно-ползунного механизма с
    тремя цилиндрами.
    \item Выполнение силового расчёта механизма для определения нагрузок в кинематических парах и
    приводного момента.
    \item Проведение уточнённого силового расчёта с учётом сил инерции и масс звеньев.
    \item Прочностной расчёт звеньев (кривошипа и шатуна) и выбор рациональных сечений и материалов.
    \item Разработка 3D-моделей звеньев и сборки механизма в САПР.
    \item Проведение МКЭ-анализа (метод конечных элементов) для проверки прочности и оптимизации
    геометрии деталей.
    \item Создание итоговой конструкторской документации и информационной MDM-модели механизма.
    \item Изготовление прототипа механизма методом 3D-печати и проверка его работоспособности.
\end{enumerate}

\paragraph{Методы и средства реализации.}
В ходе проектирования применялись методы теории механизмов и машин (ТММ), методы инженерных расчётов на прочность,
а также современные технологии компьютерного моделирования и анализа.
Работа выполнялась с использованием следующих средств:
\begin{enumerate}
    \item \textbf{Компас 3D:} для трёхмерного параметрического моделирования деталей и сборки, создания чертежей,
        проведения анализа путём МКЭ в модуле FEM;
    \item \textbf{Аналитические расчёты с использованием Python:} для выполнения кинематического и силового анализа,
        построения графиков скоростей, ускорений, моментов и реакций;
    \item \textbf{Инженерные справочники и нормативы:} СНиП 2.04.01-85*, ГОСТ 1050-2013 для определения
        расходных характеристик, выбора материалов и коэффециентов;
    \item \textbf{Технологии 3D-печати (FDM\footnote{FDM -- Fused Deposition Modeling}):} для изгототовления прототипа, демонстрирующего работоспособность
        кинематической схемы.
\end{enumerate}

\clearpage
\section{Введение}
В условиях полевой эксплуатации, удалённых работ и временного размещения практической необходимостью является
обеспечение подачи воды, в том числе в удалённые от водных ресурсов места. Текущие решения имеют ряд недостатков,
таких как зависимость от центральной системы водоснабжения, громоздкость и недостаточный напор.

Целью данной курсовой работы является проектирование и прототипирование рабочего узла переносного поршневого насоса,
отвечающего требованиям по мобильности\footnote{В данной работе под мобильностью понимается максимально допустимая
габаритно-весовая характеристики изделия для транспортировки.} и необходимому напору для свободной эксплуатации
в вышеописанных условиях.

В процессе проектирования будет решен следующий ряд задач:
\begin{itemize}
    \item метрический синтез
    \item кинематический анализ
    \item силовой анализ
    \item определение форм сечений
    \item уточненный расчёт с учетом сил инерции
    \item проверка прочности
    \item оптимизация конструкции
\end{itemize}

Проектирование проводится на основе выделенного варианта кривошипно-ползунного механизма (рисунок \ref{fig:primary_mechanism}).
\begin{figure}[h!]
    \centering
    \includegraphics[width=0.5875\linewidth]{img/primary_mechanism.png}
    \caption{Изображение исходного механизма.}
    \label{fig:primary_mechanism}
\end{figure}

В качестве идейной основы для характеристик конечного изделия были взяты характеристики умывальника по документу
СНиП 2.04.01-85*(\ref{fig:washbasin_specs})

\clearpage
\section{Качественные характеристики}

\clearpage
\section{Количественные характеристики}
Исходя из СНиП 2.04.01-85*(\ref{fig:washbasin_specs}), запишем следующие целевые характеристики:
\begin{itemize}
    \item секундный расход воды, л/с: {{ (Q * 1000) | round(1) }}
    \item секундный расход воды в кубических метрах: $Q = {{ Q | round(4) }}$
\end{itemize}

\subsection{Полезная нагрузка}
В качестве полезной нагрузки примем сопротивление жидкости (характеризуем ее давлением);
тогда, рассчитаем полезную нагрузку для подъема воды на расстояние 30 метров
\footnote{Высота 30 метров была принята для обеспечения запасной мощности насоса.}:
\begin{equation}
    P = {{ latex(P) }}
    \label{eq:liquid_pressure}
\end{equation}
, где
\begin{itemize}
    \item $P$ -- давление, Па;
    \item $\rho$ -- плотность жидкости, кг/$\text{м}^3$;
    \item $g$ -- ускорение свободного падения, \approx 9.81 м/$\text{с}^2$;
    \item $H$ -- напор, м.
\end{itemize}

Целевые значения для нашего механизма составляют:
\begin{itemize}
    \item $\rho = {{rho}}$ кг/$\text{м}^3$ (для учета загрязненной и морской воды)
    \item $H = {{H}}$ м
\end{itemize}
Таким образом:
$$
P = {{rho}} \cdot {{g}} \cdot {{H}}
$$
$$
P = {{ P_val }} \text{ Па} = {{ (P_val / 1000000) | round(3)}} \text{ МПа}
$$

При рассчете давления необходимо учитывать потери, возникающие в трубопроводе. Эти потери зависят от длины труб, их диаметра,
материала, количества изгибов и прочих факторов. Используем следующую формулу для учета этих потерь:
\begin{equation}
    \Delta P = {{latex(Delta_P) }}
    \label{eq:pipe_loss}
\end{equation}
, где
\begin{itemize}
    \item $\Delta P$ -- потери давления, Па;
    \item $\lambda_{fr}$ -- коэффициент гидравлического трения;
    \item $L_{pipe}$ -- длина трубопровода, м;
    \item $D_{pipe}$ -- диаметр трубопровода, м;
    \item $\rho$ -- плотность жидкости, кг/$\text{м}^3$;
    \item $v_{flow}$ -- скорость потока, м/с.
\end{itemize}

$$
\Delta P = {{lambda_fr}} \cdot \left(\frac{ {{ L_pipe }} }{ {{ D_pipe }} } \right) \cdot \frac{\rho \cdot {{ v_flow | round(3) }}^{2}}{2}
$$
$$
\Delta P = {{Delta_P_val | round(4)}}
$$

\subsection{Полезная мощность насоса}
Мощность, сообщаемая насосом жидкости, подаваемой в напорный патрубок (\ref{eq:full_pump_power}):
\begin{equation}
    N = {{ latex(N) }}
    \label{eq:full_pump_power}
\end{equation}
, где
\begin{itemize}
    \item $N$ -- полная мощность насоса, Вт;
    \item $Q$ -- расход жидости, $\text{м}^3$/с;
    \item $\rho$ -- плотность жидкости, кг/$\text{м}^3$;
    \item $g \approx {{g}}$ м/$\text{с}^2$;
    \item $H = {{H}}$ м.
\end{itemize}
$$
N = {{ Q | round(4) }} \cdot {{ H }} \cdot {{ rho }} \cdot {{ g }}
$$
$$
N = {{ N_val | round(4)}}
$$
Для получения мощности двигателя насоса, необходимо учесть КПД, возникающий в реальной конструкции;
в случае нашего изделия, КПД $\eta_{\Sigma} \in \left[ 60\%, 95\% \right]$. Возьмем нижний порог,
$\eta_{\Sigma} = 60\%$:
\begin{equation}
    N_{used} = {{ latex(N_used) }}
    \label{eq:pump_power}
\end{equation}
$$
N_{used} = \frac{ {{ N_used_val | round(4) }} }{ {{eta_Sigma | round(2)}} }
$$
$$
N_{used} = {{N_used_val | round(3)}}
$$

\subsection{Пульсация}
Пульсация -- периодические колебания скорости или расхода жидкости, проходящий через элемент гидравлической системы.
Поршневой насос даёт неравномерную подачу, для выравнивания применяем многократность.
Определим следующие функции для нахождения пульсации для нескольких цилиндров:
\begin{align*}
    & q(\phi) = \text{max}(0, sin(x)) \\
    & q_\Sigma(\phi, n) = \sum_{k = 0}^{n - 1}q(x + \frac{2k\pi}{n}) \{\phi \ge 0\} \\
    & \tilde{q}(\phi, n) = \frac{q_\Sigma(\phi, n)}{q_\Sigma(\frac{n\pi}{2})}
\end{align*}
, где \begin{itemize}
    \item $\phi$ -- фаза цилиндра, рад.;
    \item $n$ -- количество цилиндров в насосе;
    \item $q(\phi)$ -- пульсация одного цилидра
        \footnote{Ограничение на положительные значения введено так как в данном случае отрицательные значения не имеют физического смысла}
        , относительный моментальный расход (\%);
    \item $q_\Sigma(\phi, k)$ -- суммарная пульсация всех цилиндров, относительный моментальный расход (\%);
    \item $\tilde{q}(\phi, k)$ -- нормированная суммарна пульсация всех цилиндров, относительный моментальный расход (\%).
\end{itemize}
Рассмотрим графики пульсации для 1, 3, 5 и 7 (рисунки \ref{fig:pulse_1}, \ref{fig:pulse_3}, \ref{fig:pulse_5} и \ref{fig:pulse_7} соответственно)
цилиндров -- нечётное количество цилиндров дает более гладкий поток, уменьшает
вибрации и позволяет реализовать более простую конструкцию, все эти факторы важны для нашего изделия.

Заметим, что при одном цилиндре имеются периоды простоя с резкими пиками -- пульсация слишком сильная;
при 5 и 7 цилиндрах пульсация выравнивается, но конструкция усложняется и на масштабе изделия является избыточной.


\begin{figure}[h!]
    \centering
    \includegraphics[width=\linewidth]{img/gen/pulse_{{ i }}.png}
    \caption{Графики нормированной пульсации для {{ i }} цилиндра(-ов)}
    \label{ fig:pulse_{{ i }} }
\end{figure}


\clearpage
\section{Рассчетная часть}
\subsection{Метрический синтез}
Для метрического синтеза составим кинематическую схему механизма, обозначив характерные точки механизма (рисунок \ref{fig:cinematic_schema}):

\begin{figure}[h!]
    \centering
    \includegraphics[width=\linewidth]{img/cinematic_schema.png}
    \caption{Кинематическая схема механизма}
    \label{fig:cinematic_schema}
\end{figure}

При помощи кинематической схемы составим траектории точек и выразим уравнения.
Возьмем за начало системы координат точку $A$, то есть ось кривошипа.
Пусть $R$ -- длина кривошипа. Опишем точки:
\begin{itemize}
    \item Точка $A$:
        \begin{align*}
            x_A(\phi) &= 0 \\
            y_A(\phi) &= 0
        \end{align*}
    \item Точка $B$:
        \begin{align*}
            x_B(\phi) &= R \cdot \cos{\phi} \\
            y_B(\phi) &= R \cdot \sin{\phi}
        \end{align*}
    \item Точка $C$:
        \begin{align*}
            x_C(\phi) &= R \cdot \cos{\phi} + \sqrt{L^2 - R^2 \cdot \sin^2{\phi}} \\
            y_C(\phi) &= 0
        \end{align*}
    \item Точка $D$:
        \begin{align*}
            x_D(\phi) &= 0 \\
            y_D(\phi) &= 0
        \end{align*}
\end{itemize}

Характерные точки: $E$ и $F$ - середины кривошипа и шатуна (звеньев 1 и 2 соответственно):
\begin{itemize}
    \item Точка $E$:
        \begin{align*}
            x_E(\phi) &= R \cdot \frac{1}{2}\cos{\phi} \\
            y_E(\phi) &= R \cdot \frac{1}{2}\sin{\phi}
        \end{align*}
    \item Точка $F$:
        \begin{align*}
            x_F(\phi) &= R \cdot \cos{\phi} + \frac{1}{2}\sqrt{L^2 - R^2 \cdot \sin^2{\phi}} \\
            y_F(\phi) &= \frac{1}{2}R \cdot \sin{\phi}
        \end{align*}
\end{itemize}

Из уравнений $x_C(\phi)$ и $x_F(\phi)$ имеем, что $R \le L$, то есть длина кривошипа не больше его длины.

\subsection{Длины}
Вычислим длины компонентов. За значение $d$ возьмем сантихнический стандарт -- $1 \text{ дюйм } \approx 25.4 \text{ мм}$ и $\frac{3}{4}$ дюйма:
\begin{align*}
    & A = \frac{Q}{2Rzn\eta_v} = {{latex(A)}} \\
    & R = {{latex(R)}} \\
    & S = 2 \cdot R = {{latex(S)}} \\
    & \lambda_{len} = \frac{L}{R}, \text{ пусть } \lambda = 3 \\
    & L = \lambda_{len} = {{ latex(L) }} \\
    % & v_{\text{max}} \approx \omega \cdot R \\
    & F = P A = {{latex(F)}}
\end{align*}
Подставим численные значения:
\begin{itemize}

\item Возьмем $d  = {{ (d * 1000) | round(2) }} \text{ мм}$:
\begin{align*}
    & A = {{ (as_fun('d', A_val)(d) * 10000) | round(4)}}\text{ мм}^{-2} \\ 
    & R = {{ (as_fun('d', R_val)(d) * 1000) | round(4)}}\text{ мм} \\ 
    & S = {{ (as_fun('d', S_val)(d) * 1000) | round(4)}}\text{ мм} \\ 
    & L = {{ (as_fun('d', L_val)(d) * 1000) | round(4)}}\text{ мм} \\ 
    & F = {{  as_fun('d', F_val)(d) | round(4)}}\text{ Н} \\
\end{align*}

\end{itemize}

В случае $d = 25.4$мм механизм получается слишком коротким и слишком широким -- возьмем $d = \frac{3}{4} \cdot 25.4 = 19.05$мм.

\clearpage
\section{Приложения}
\begin{figure}[h!]
    \centering
    \includegraphics[width=\linewidth]{img/washbasin_specs.png}
    \caption{Характеристики умывальника согласно документу СНиП 2.04.01-85*}
    \label{fig:washbasin_specs}
\end{figure}

\begin{figure}[h!]
    \centering
    \includegraphics[width=\linewidth]{img/pump_specs.png}
    \caption{TODO}
    \label{fig:pump_specs}
\end{figure}

\end{document}
